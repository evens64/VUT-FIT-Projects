% !TeX encoding = UTF-8
\documentclass[a4paper, 11pt]{article}

% packages
\usepackage[czech]{babel}
\usepackage[utf8]{inputenc}
\usepackage[left=2cm, text={17cm, 24cm}, top=3cm]{geometry}
\usepackage[IL2]{fontenc}
\usepackage{bookman}
\usepackage[numbers]{natbib}
\usepackage[hidelinks]{hyperref}


\bibliographystyle{bib/czplain}

\renewcommand{\bibsection}{\section{\bibname}}

\newcommand{\tab}[1][1cm]{\hspace*{#1}}

\newenvironment{envtt}{\ttfamily}{\par}


%titulní strana, obsah, logické strukturování textu, výcuc relevantních informací z nastudované literatury, popis zajímavějších pasáží implementace, sekce o testování, bibliografie, popisy k řešení bonusových zadání

\begin{document}

\begin{titlepage}
\begin{center}

\textsc{{\Huge Vysoké učení technické v~Brně}\\\medskip{\huge Fakulta informačních technologií}}

\vspace{\stretch{0.382}}

{\LARGE Síťové aplikace a správa sítí}\\\smallskip{\Huge Programování síťové služby \\\smallskip LDAP server}

\vspace{\stretch{0.618}}

{\Large \today \hfill Jakub Frýz}

\end{center}
\end{titlepage}

\tableofcontents
\pagebreak

\section{Úvod}

\subsection{Zadání}

Úkolem bylo vytvořit jednoduchý LDAP server, který bude načítat dotazy od LDAP klientů a vyhledávat odpovědi v lokální textové databázi.

Server nemusí podporovat češtinu či jiné národní znaky. Server nevyžaduje autentizaci, stačí podpora LDAPv2, který řeší pouze vyhledávání. Ostatní zprávy server ignoruje, popřípadě odpoví, že danému příkazu nerozumí.

Pracovní databáze serveru má formát textového souboru CSV (comma separated values se středníkem jako oddělovačem), který obsahuje tři položky - Jméno (\texttt{cn}, \texttt{CommonName}), Login (\texttt{uid}, \texttt{UserID}), Email (\texttt{mail}).

Co se týče TLV filter, tak nad pracovní databází připadá v úvahu podpora \texttt{and}, \texttt{or}, \texttt{not}, \texttt{equalityMatch} a \texttt{substrings}.\footnote{Pozn.: zadání bylo převzato a upraveno z WIS}

\pagebreak


\section{Popis}

V následujících podkapitolách se budu snažit přiblížit stručně protokol LDAP.

\subsection{Protokol LDAP}

\textbf{LDAP} (\textit{Lightweight Directory Access Protocol}) je internetový protokol pro přístup k adresářovým službám. Je odlehčenou verzí předchozího protokolu \texttt{X.500}.

Systém LDAP je definován množinou čtyř modelů: Informační, jmenný, funkční a bezpečnostní model. 

\begin{itemize}
	\item \textbf{Informační model}
	\begin{itemize}
		\item popisuje uložení informací v adresáři
		\begin{itemize}
			\item záznamy = souhrn atributů (dvojice název-hodnota)
		\end{itemize}
		\item definuje typy dat a operace nad nimi (porovnání,\dots)
	\end{itemize}

	\item \textbf{Jmenný model}
	\begin{itemize}
		\item popisuje strukturu adresáře
		\item tvoří stromovou strukturu DIT (Directory Information Tree)
	\end{itemize}
	
	\item \textbf{Funkční model}
	\begin{itemize}
		\item zabývá se přístupem k datům a operacemi nad nimi (vyhledávání,\dots)
	\end{itemize}

	\item \textbf{Bezpečnostní model}
	\begin{itemize}
		\item zabývá se zabezpečením dat
		\item definuje práva pro vkládání a modifikaci dat adresáře
		\item stará se o autentizaci uživatele (SASL)
	\end{itemize}

\end{itemize}

Protokol LDAP se používá například pro vyhledávání e-mailové adresy u poštovního klienta nebo pro autentizaci uživatele.

\pagebreak


\subsection{Standard ASN.1}

\textbf{ASN.1} (\textit{Abstract Syntax Notation One}) je prostředek pro popis datových struktur pro reprezentaci, kódování, přenos, ukládání a dekódování dat v telekomunikacích, počítačových sítích a informatice. Poskytuje soubor pravidel umožňujících popsat strukturu objektů způsobem nezávislým na konkrétním hardwarovém řešení. \cite{wiki_asn1}

\begin{center}
\begin{minipage}{.7\textwidth}
\begin{flushleft}
\begin{envtt}
	LDAPMessage ::= SEQUENCE \{	\\
\tab	messageID       MessageID, \\
\tab	protocolOp      CHOICE \{ \\
\begin{tabbing}
	\tab\tab xxxxxxxxxxxxxxxx\quad	\= xxxxxxxxxxxxxxxx			\kill
	\tab\tab bindRequest    \> BindRequest, \\
	\tab\tab bindResponse   \> BindResponse, \\
	\tab\tab unbindRequest  \> UnbindRequest, \\
	\tab\tab searchRequest  \> SearchRequest, \\
	\tab\tab searchResEntry \> SearchResultEntry, \\
	\tab\tab searchResDone  \> SearchResultDone, \\
	\tab\tab searchResRef   \> SearchResultReference, \\
	\tab\tab modifyRequest  \> ModifyRequest, \\
	\tab\tab modifyResponse \> ModifyResponse, \\
	\tab\tab addRequest     \> AddRequest, \\
	\tab\tab addResponse    \> AddResponse, \\
	\tab\tab delRequest     \> DelRequest, \\
	\tab\tab delResponse    \> DelResponse, \\
	\tab\tab modDNRequest   \> ModifyDNRequest, \\
	\tab\tab modDNResponse  \> ModifyDNResponse, \\
	\tab\tab compareRequest \> CompareRequest, \\
	\tab\tab compareResponse\> CompareResponse, \\
	\tab\tab abandonRequest \> AbandonRequest, \\
	\tab\tab extendedReq    \> ExtendedRequest, \\
	\tab\tab extendedResp   \> ExtendedResponse \},
\end{tabbing}
\tab		controls       [0] Controls OPTIONAL \}
\end{envtt}
\end{flushleft}
\end{minipage}
\end{center}
\bigskip

Výše je ukázka z LDAP ASN.1 gramatiky \cite{rfc_2251}.

\pagebreak


\subsection{Kódování BER}

\textbf{BER} (\textit{Basic Encoding Rules}) se používá pro kodóvání \texttt{ASN.1}. BER patří k TLV formátům (TLV = type-length-value = typ-délka-hodnota).
\bigskip

Pole \texttt{type} určuje význam hodnoty. Má následující strukturu:
	
\begin{table}[h]
	\centering
	\begin{tabular}{|c|c|c|c|c|c|c|c|}
		\hline
		8            & 7            & 6   & 5    & 4    & 3    & 2    & 1   \\ \hline
		\multicolumn{2}{|c|}{Třída} & P/C & \multicolumn{5}{c|}{Číslo tagu} \\ \hline
	\end{tabular}
\end{table}

Třída určuje, kde je typ nadefinován (typy třídy \texttt{00} jsou nadefinované přímo standardem ASN.1). P/C rozlišuje, zda-li je hodnota jednoduchá nebo složená (obsahuje jiné položky). A zbytek je číslo tagu, který určuje konkrétní tag (2 = \texttt{integer}, 10 = \texttt{enumerated}, 16 = \texttt{sequence}).
\bigskip


Např.: Sekvence oktetů \texttt{02 01 01} znamená: integer délky 1 a hodnoty 1


\pagebreak


\section{Implementace}

Aplikace byla napsána v jazyce C++. Část komunikace byla vytvořena pomocí projektu z minulého roku z IPK.

Při tvorbě mi hodně pomohla stránka \href{http://en.cppreference.com/w/}{cppreference.com}.


\subsection{Signál SIGINT}

Pomocí \texttt{sigaction} odchytávám \texttt{SIGINT} signál. Ten teď namísto toho, aby natvrdo ukončí aplikaci, ukončí smyčku pro nová připojení, ukončí sokety a uzavře soubor.

\subsection{Komunikace}

K vytvoření komunikace používám funkce \texttt{socket()}, \texttt{bind()}, \texttt{listen()} a \texttt{accept()} v tomto pořadí. Funkce \texttt{accept()} slouží k připojení klienta, a proto je použita cyklu.

\subsection{Neblokující soket}

Tohoto jsem dosáhl pomocí funkce \texttt{select()}, kterou jsem použil i minulý rok v projektu do IPK.

Jedinou nevýhodou mého řešení je zvýšená zátěž na procesor.

\subsection{Procesy}

Po připojení klienta k serveru se pomocí fce \texttt{fork()} vytvoří tzv. 'dětský proces', který zdědí informace z paměťového prostoru 'rodičovského procesu' a následně si vytvoří svůj vlastní paměťový prostor. Díky tomu neinterferuje s jinými procesy a přesto přístup k databázi.

\subsection{LDAP zprávy}

Následně se musí dekódovat požadavky z klienta. 

Moje funkce se pouze podívá na určitá místa požadavku a přečte si jejich hodnotu (přesněji řečeno dvě místa -- identifikátor zprávy a typ zprávy).
Následuje zakódování odpovědi. Celou zprávu mám natvrdo zadanou, jen do ní zapíšu identifikátor zprávy.
Toto platí pro požadavky typu \texttt{BindRequest} a \texttt{UnbindRequest} a odpovědi typu \texttt{BindResponse} a \texttt{SearchResultDone}.

Pro požadavky typu \texttt{SearchRequest} mám vytvořen stavový automat, který vyčte všechny potřebné informace z různě dlouhých zpráv.

A pro odpovědi typu \texttt{SearchResultEntry} mám funkce pro vytvoření celé zprávy podle toho, co bylo nalezeno v databázi.

\subsection{Vyhledávání v DB}

Databáze ve formátu CSV se načítá do proměnné typu \texttt{ifstream}. Tento typ je pouze pro čtení.

Databázi procházím po jednom od začátku. Využívám k tomu funkci \texttt{getline()} pro načtění řádku databáze a funkci \texttt{strtok()} pro rozdělení atributů podle znaků ';'.

Pokud byl \texttt{SearchRequest} typu \texttt{equalityMatch}, používám fce \texttt{strcmp}, a pro \texttt{SearchRequest} typu \texttt{substrings} používám metodu \texttt{find()}, kterou má typ \texttt{string}.


\pagebreak


\section{Testování}

Testování převážně probíhalo na virtuálním stroji na mém počítači. Na tomto virtuálním stroji jsem měl nainstalovaný OS Kubuntu 17.10.

Pro účely testování jsem si vytvořil vlastní databázi ve formátu CSV, která je stejná jako ukázková ve WIS, ale neobsahuje diakritická znaménka. Tato databáze je přibalena v odevzdaném archívu pod názvem \texttt{db.csv}. Symbol '@' jsem v e-mailových adresách ponechal, protože se mi na virtuálním stroji zobrazují správně, ale na serveru \texttt{eva} se mi e-mailové adresy zobrazují jako shluk symbolů. Za to může pravděpodobně jiné kódování řetězce ze strany klienta.
\bigskip

Značná část pomocných výstupu použitých k testování byla z odevzdané verze vymazána.

\begin{table}[h]
	\centering
	\label{mypc}
	\begin{tabular}{|l|l|}
		\hline
		\textbf{CPU}       & Intel i5-6600K 3.3GHz                        \\ \hline
		\textbf{GPU}       & MSI R9 390X GAMING 8G 8GB                    \\ \hline
		\textbf{RAM}       & Kingston HyperX Fury Black 16GB DDR4 2133MHz \\ \hline
		\textbf{DISK \#01} & Samsuing SD 850 EVO 250GB                    \\ \hline
		\textbf{DISK \#02} & WD Green EZRX 2TB (5400rpm)                  \\ \hline
		\textbf{DISK \#03} & Samsuing SD 850 EVO 500GB                    \\ \hline
		\textbf{OS}        & Windows 10 Education 1703                    \\ \hline
	\end{tabular}
	\caption{HW konfigurace testovacího PC}
\end{table}

Virtuální stroj běžel na disku č. 3.
\bigskip

Testování proběhlo úspěšně i na serveru \texttt{merlin}, ke kterému jsem se připojoval ze serveru \texttt{eva}.

\pagebreak


\section{Omezení}

\begin{itemize}
	\item filtry \texttt{and} a \texttt{or} nejsou podporovány

	\item při \texttt{SearchResultEntry} nejsou ošetřeny některé chyby (aplikace v tom případě odešle klientu \texttt{SearchResultDone} s návratovou hodnotou \texttt{0} a žádný \texttt{SearchResultEntry})
	
	\item ignoruje otázky na název objektu
	
	\item pokud je TLV filtr \texttt{SearchRequestu} typu \texttt{present}, server vypíše prvních \texttt{SIZE\_LIMIT} záznamů (u mne to funguje, na \texttt{evě} klient vrátí chybu)
\end{itemize}



\pagebreak


\nocite{*}
\bibliography{dokumentace}

\end{document}
