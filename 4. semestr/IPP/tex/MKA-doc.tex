\documentclass[a4paper, 11pt]{article}

\usepackage[utf8]{inputenc}
\usepackage[czech]{babel}
\usepackage[left=2cm, text={17cm, 24cm}, top=2.5cm]{geometry}
\usepackage[IL2]{fontenc}
\usepackage{times}	% Times New Roman font
\usepackage{textcomp}

\begin{document}

\begin{flushleft}

Dokumentace úlohy MKA: Minimalizace konečného automatu v Python 3 do IPP 2016/2017

Jméno a příjmení: Jakub Frýz

Login: xfryzj01

\noindent\rule{\textwidth}{0.4pt}
	
\end{flushleft}

\section*{Úvod}

Následující text popisuje proces skriptu

\section{Vstupní argumenty}

Vstupní argumenty se zpracovávají pomocí \texttt{argparse}. Kontroluje se:

\begin{itemize}
	
	\item je-li zadán argument \texttt{--help} a nic navíc:
	\begin{itemize}
		\item[\textbf{ano}] vypíše se nápověda
		\item[\textbf{ne}] CHYBA
	\end{itemize}

	\item jsou-li zadány oba argumenty \texttt{--find-non-finishing} (\texttt{-f}) a \texttt{--minimize} (\texttt{-m}) $\rightarrow$ CHYBA

	\item je-li zadán argument \texttt{--input}:
	\begin{itemize}
		\item[\textbf{ano}] načtou se data ze souboru
		\item[\textbf{ne}] čte se ze standardního vstupu \texttt{stdin}
	\end{itemize}

	\item je-li zadán argument \texttt{--output}:
	\begin{itemize}
		\item[\textbf{ano}] data se uloží do souboru
		\item[\textbf{ne}] data se vytisknou na standardní výstup \texttt{stdout}
	\end{itemize}

	\item je-li zadán argument \texttt{--case-insensitive} (\texttt{-i}) $\rightarrow$ po načtení souboru se celý soubor převede na malá písmena

\end{itemize}

\section{Zpracování vstupního souboru}

Vstupní soubor zpracovávám převážně za pomoci regexů. Na začátku odstraním komentáře a všechny bílé znaky, co nejsou mezi \textquotesingle\textquotesingle. Následně pak pomocí funkce \texttt{glue()} převedu celý automat na jednořádkový \texttt{string} tzn. že zbylé bílé znaky byly převedeny na jejich \uv{ne bílou formu} (\verb|\n|, \verb|\t|, \verb|\r|). Na takto vzniklý řetězec se zavolá funkce \texttt{process()}, která za pomoci tohoto regexu
\begin{center}
	\verb|^\(\{(.*)\},\{(.*)\},\{(.*)\},(.*),\{(.*)\}\)$|
\end{center}
získá z onoho řetězce pětici, která definuje automat. Ta je následně pomocí dalších regexů rozkouskována na jednotlivé stavy, znaky, atd.. Ve výsledku funkce vrátí 4 \texttt{listy} (stavy, abeceda, pravidla, konečné stavy) a jeden \texttt{string} (počáteční stav). Těchto pět položek je po návratu z funkce uloženo do dalšího \texttt{listu} do proměnné jménem \texttt{automat}.

\pagebreak

Teď je v proměnné \texttt{automat} uložen konečný automat, ale nejsme si jisti, zda-li je dobře specifikovaný. Na to je funkce \texttt{specify()}, která zkontroluje, zda-li automat splňuje podmínky:
\begin{itemize}
	\item neobsahuje $\varepsilon$-přechody
	\item neobsahuje nedeterminismus
	\item neobsahuje nedostupné a neukončující stavy
\end{itemize}
Pokud podmínky splňuje funkce navrátí hodnotu \texttt{True}, jinak \texttt{False}. Pokud vrátí \texttt{False}, skript se ukončí s návratovou hodnotou \texttt{62}.

Tímto by mělo být potvrzeno, že zadaný automat je dobře specifikovaný.

Pokud byl zadán argument \texttt{--find-non-finishing} (\texttt{-f}) a nebo \texttt{--minimize} (\texttt{-m}) provedou se odpovídající funkce.

Následuje už jenom výpis automatu. Ten se provede buď na standardní výstup nebo do souboru, byl-li zadán argument \texttt{--output}.

\section{Nalezení neukončujícího stavu}

K tomuto účelu zde slouží funkce \texttt{find\_non\_finishing()}, která není nic jiného než upravená část funkce \texttt{specify()} s jediným rozdílem, že vrací název neukončujícího stavu.

\section{Minimalizace konečného automatu}

POZN: minimalizaci se mi nepodařilo do konce pořádně zprovoznit.
\\ \\
Minimalizace automatu má vytvořit minimální možnou verzi zadaného automatu. To provede funkce \texttt{minimize()}. Vytvoří se dvě množiny stavů (ty koncové a ty ostatní). Následně se prochází cyklem přes znaky abecedy a hledají se pravidla, která vycházejí ze stavů jedné a pak i druhé množiny samostatně. Pokud všechna nalezená pravidla pro daný znak i množinu stavů končí ve stavech, která jsou všechny součástí jedné množiny, nic se neděje, pokud ovšem se stane, že některé jsou z druhé množiny, provede se štěpení a štěpí se tak dlouho, dokud se už nedá více štěpit.

Pak dojde ke spojování stavů. Projdou se všechna pravidla znovu a pokud se najdou taková pravidla, která přes stejný znak vycházejí ze stavů z jedné množiny a končí v jiné či stejné, lze tyto pravidla nahradit spojenou verzí.

\section*{Závěr}

Tenhle projekt byl pro mne mnohem těžší než předchozí. Jedním z důvodů může příliš volný styl kódu jazyka Python3. Osobně preferuji, když jsem syntaxí jazyka trochu \uv{omezován} jako například znaky \{ a \} pro funkce, cykly, podmínky, atd.. Osobně mi tento styl přijde přehlednější. Ale zde už záleží na vkusu člověka, který ten jazyk používá. I tak jsem se mnohé přiučil, ať už z jazyka Python3, ale i tím, že jsem si mohl vyzkoušet algoritmizaci už známých procesů z předmětu IFJ.

\end{document}
